%% start of file `template-zh.tex'.
%% Copyright 2006-2013 Xavier Danaux (xdanaux@gmail.com).
%
% This work may be distributed and/or modified under the
% conditions of the LaTeX Project Public License version 1.3c,
% available at http://www.latex-project.org/lppl/.


\documentclass[11pt,a4paper,sans]{moderncv}  
\usepackage[utf8]{inputenc}                  
\usepackage{CJKutf8}
\usepackage{info}
% moderncv 主题
                     

\setlength{\hintscolumnwidth}{3cm}           % 如果你希望改变日期栏的宽度
 

% 个人信息
\name{黄}{峰达}

%\title{简历题目 (可选项)}                     % 可选项、如不需要可删除本行
\address{陕西省西安市太白南路168号}{710065}            % 可选项、如不需要可删除本行
\phone[mobile]{+18209219631}              % 可选项、如不需要可删除本行
% 可选项、如不需要可删除本行
\email{h@phodal.com}                    % 可选项、如不需要可删除本行
\homepage{www.phodal.com}                  % 可选项、如不需要可删除本行
\extrainfo{Power by \LaTeX  }                 % 可选项、如不需要可删除本行
\photo[64pt][0.2pt]{picture}                  % ‘64pt’是图片必须压缩至的高度、‘0.4pt‘是图片边框的宽度 (如不需要可调节至0pt)、’picture‘ 是图片文件的名字;可选项、如不需要可删除本行
\begin{document}
\begin{CJK}{UTF8}{gbsn}                       % 详情参阅CJK文件包
\maketitle

\section{{教育经历}}
\tlcventry{2010.9}{2014.6}{{本科在读}}{电子信息工程}{西安文理学院}{西安}{}

\tllabelcventry{2012.8}{2012.12}{2012.12}{挑战杯科技制作大赛特等奖}{西安文理学院}{}{}{
\begin{tightitemize}
	\item 名称:《基于Android与Arduino的远程智能家居控制系统》 
	\item 职责: Android客户端编写、服务端编程、网站制作及运营、项目规划
\end{tightitemize}}
\tllabelcventry{2012.10}{2013.6}{2013.6}{第九届 陕西省大学生课外学术科技作品竞赛二等奖}{}{}{}{
\begin{tightitemize}
	\item 名称:《远程控制的智能云家居系统》 
	\item 职责: Android客户端编写、服务端编程、网站制作及运营、项目规划、文档编写
\end{tightitemize}}
\tllabelcventry{2013.1}{2013.7}{2013.7}{第八届全国大学生“飞思卡尔”杯}{西部赛区 电磁组三等奖}{}{}{}

\section{专业技能}
\subsection{开发}
\cvcomputer{芯片}{Atmega family, 51, STM32, MSP430, Freescale K60, Freescale XS128, ARM,}
           {开发环境}{IAR for ARM, Keil C51, ADS, Arduino IDE, Android Sudio,Eclipse,Energia}

\subsection{工具}

\cvcomputer{办公}{iWork, OpenOffice/LibreOffice, Microsoft Office, \LaTeX, Graphviz}
           {操作系统}{Mac OS X, GNU/Linux(penSUSE, Centos), Windows}
\cvcomputer{Design}{ Photoshop, Dot, Texworks}
           {编辑器}{Emacs, VIM, Notepad++, Eclipse, Aptana}

\cvcomputer{EOS}{$\mu$C/OS $\Pi$,openWRT Linux,Linux for ARM}
		{其他}{openCV, TTS, Raspberry Pi, MATLAB,}

\section{{项目经历}}
\tllabelcventry{2012.10}{2013.6}{}{远程控制的智能云家居系统}{项目负责人}{}{}{
\begin{tightitemize}
        \item 关键技术:Django, RESTful API,Android ADK,物联网, Ajax,XBEE
        \item 编程语言:Python,Java,C/C++,Processing
        \item 采用XBee来实现无线传感器网络
        \item wiki主页:{\httplink{api.phodal.com/wiki/}}
\end{tightitemize}}

\subsection{其他经历}
\tlcventry{2010.12}{0}{其他课程项目、研究}{}{}{}{
\begin{tightitemize}
 \item 使用QT4完成移动蜂窝模型通信系统构建
 \item 使用Node.j和WebSocket实现了简易在线聊天系统  {\httplink{socket.phodal.com}}
 \item 使用flex和bison ,antlr完成了简单的翻译器
 \item 使用Python、Django、 Bootstrap维护的个人网站 {\httplink{www.phodal.com}}
 \item 使用Freescale K60实现寻线小车,使用Freescale XS128实现飞思上尔电磁车
 \item 个人博客、SEO优化、微数据(Microdata) {\httplink{blog.phodal.com}}
 \item 使用LFS编译自己的Linux系统
 \item 移植GCC for ARM到Android G1 {\httplink{blog.csdn.net/phodal}}
 \item 使用OpenWRT Linux实现路由器上运行Python,运行RESTful服务
\end{tightitemize}}

\section{语言技能}
\cvitemwithcomment{英语}{}{精通读写}
\cvitemwithcomment{闽南语}{}{精通听说}
\section{个人信息}
\cvitem{特长}{\small 计算机相关,互联网相关,网站创作、设计及运维,嵌入式系统设计,架构设计, DIY,设计,开源}
\cvitem{职业技能}{\small 精通GNU/Linux,能在不同的平台编程,精通Emacs,熟练使用Vim,同时能使用Eclipse/及Visual studio编程,熟练使用Android与单片机通讯,不同单片机设备间的通讯。精通不同平台的排版及文档创作,精通使用office,以及tex/latex/miktex/生成文档,dot生成文档,系统构架分析及设计。}
\renewcommand{\listitemsymbol}{-}             % 改变列表符号

% 来自BibTeX文件但不使用multibib包的出版物
%\renewcommand*{\bibliographyitemlabel}{\@biblabel{\arabic{enumiv}}}% BibTeX的数字标签
\nocite{*}
\bibliographystyle{plain}
\bibliography{publications}                    % 'publications' 是BibTeX文件的文件名

% 来自BibTeX文件并使用multibib包的出版物
%\section{出版物}
%\nocitebook{book1,book2}
%\bibliographystylebook{plain}
%\bibliographybook{publications}               % 'publications' 是BibTeX文件的文件名
%\nocitemisc{misc1,misc2,misc3}
%\bibliographystylemisc{plain}
%\bibliographymisc{publications}               % 'publications' 是BibTeX文件的文件名

\clearpage\end{CJK}
\end{document}


%% 文件结尾 `template-zh.tex'.
