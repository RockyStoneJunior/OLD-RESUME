%% start of file `template-zh.tex'.
%% Copyright 2006-2013 Xavier Danaux (xdanaux@gmail.com).
%
% This work may be distributed and/or modified under the
% conditions of the LaTeX Project Public License version 1.3c,
% available at http://www.latex-project.org/lppl/.


\documentclass[11pt,a4paper,sans]{moderncv}  
\usepackage[utf8]{inputenc}                  
\usepackage{CJKutf8}
\usepackage{info}                 

\setlength{\hintscolumnwidth}{3cm}           
\name{黄}{峰达}
\title{Phodal}                      
\address{陕西省西安市太白南路168号}{710065}            
\phone[mobile]{+18209219631}              
\email{h@phodal.com}                    
\homepage{www.phodal.com}                   
\extrainfo{Power by \LaTeX  }                   
\photo[64pt][0.2pt]{avatar}                   

\begin{document}
\begin{CJK}{UTF8}{gbsn}                        
\maketitle

\section{{教育背景}}
\cventry{2010.9-2014.6}{{本科}}{电子信息工程}{西安文理学院}{西安}{}

\section{{工作经历}}
\cventry{2013.12-2014.6}{{实习}}{软件开发工程师}{ThoughtWorks}{西安}{}{}
\cventry{2014.7-今}{{在职}}{软件开发工程师}{ThoughtWorks}{西安}{}{}

\section{{项目经历}}
\cventry{2013.12-今}{澳洲某房地产搜索网站}{开发人员}{}{}{
\begin{tightitemize}
        \item 项目描述: 维护、开发澳洲某房地产搜索网站桌面版、移动版网站,并使用新的架构对现有的系统重构。
        \item 编程语言: Java, JavaScript, Ruby, Node.js
        \item 项目技术:  Spring MVC, FAST/Solr, Backbone, React, 微服务, AWS, Grunt
        \item 主要贡献: 1. 开发新首页。 2. 使用React开发新详情页。
\end{tightitemize}}

\section{技能}

\subsection{语言}
\cvcomputer{熟练}{JavaScript, Java, Node.js}
           {熟悉}{Python, Ruby}

\subsection{框架}           
\cvcomputer{熟练}{Backbone, Spring, Django}           
           {熟悉}{Angluar, React}
		
\section{社区}
\cvitemwithcomment{Github}{ \httplink{github.com/phodal}}{}
\cvitemwithcomment{InfoQ}{ http://www.infoq.com/cn/author/黄峰达 }{ 编辑}
\cvitemwithcomment{CSDN}{ \httplink{blog.csdn.net/phodal}}{博客专家,头条主编}
\cvitemwithcomment{SegmentFault}{ \httplink{segmentfault.com/u/phodal}}{Top Writer}

\subsection{其他}           
\cventry{2014}{PACKT 出版社《Learning Internet of Things》}{专家Review}{}{}{}
\cventry{2015}{Manning出版社 某IoT书籍}{专家Review}{}{}{}

\renewcommand{\listitemsymbol}{-}             % 改变列表符号

% 来自BibTeX文件但不使用multibib包的出版物
%\renewcommand*{\bibliographyitemlabel}{\@biblabel{\arabic{enumiv}}}% BibTeX的数字标签
\nocite{*}
\bibliographystyle{plain}
\bibliography{publications}                    % 'publications' 是BibTeX文件的文件名

% 来自BibTeX文件并使用multibib包的出版物
%\section{出版物}
%\nocitebook{book1,book2}
%\bibliographystylebook{plain}
%\bibliographybook{publications}               % 'publications' 是BibTeX文件的文件名
%\nocitemisc{misc1,misc2,misc3}
%\bibliographystylemisc{plain}
%\bibliographymisc{publications}               % 'publications' 是BibTeX文件的文件名

\clearpage\end{CJK}
\end{document}


%% 文件结尾 `template-zh.tex'.
