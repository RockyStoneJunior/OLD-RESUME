%% start of file `template-zh.tex'.
%% Copyright 2006-2013 Xavier Danaux (xdanaux@gmail.com).
%
% This work may be distributed and/or modified under the
% conditions of the LaTeX Project Public License version 1.3c,
% available at http://www.latex-project.org/lppl/.


\documentclass[11pt,a4paper,sans]{moderncv}  
\usepackage[utf8]{inputenc}                  
\usepackage{CJKutf8}
\usepackage{info}                 

\setlength{\hintscolumnwidth}{3cm}           
\name{黄}{峰达}
\title{Phodal Huang}                      
\phone[mobile]{+086 18209219631}              
\email{h@phodal.com}                    
\homepage{https://www.phodal.com}                   
\photo[64pt][0.2pt]{avatar}                   

\begin{document}
\begin{CJK}{UTF8}{gbsn}                        
\maketitle

\section{{工作经历}}
\cventry{2013.12-2014.6}{{实习}}{软件开发工程师}{ThoughtWorks(西安)}{}{}{}
\cventry{2014.7-今}{{在职}}{软件开发工程师}{ThoughtWorks(西安、深圳)}{现在深圳}{}{}

\section{{项目经历}}
\cventry{2013.12-2016.04}{ThoughtWorks - 澳洲某大型房地产搜索网站}{}{开发人员}{技术负责人}{
\begin{tightitemize}
        \item 项目描述: 开发网站的桌面版、移动版、API、服务组件,并用微服务架构与响应式设计对系统重构。
        \item 语言: Java、JavaScript、Ruby、Sacala
        \item 技术栈:  Spring MVC、Node.js、ElastichSearch、Backbone、React、微服务、AWS
        \item 主要贡献: 作为开发人员,负责网站主站维护、移动网站开发。作为技术负责人,负责对网站进行重构,并采用新架构进行设计。并对新人进行指导和培训,协助团队进行更好的进行敏捷开发及软件工程实践。
\end{tightitemize}}
\cventry{2016.04-今}{ThoughtWorks - 某大型国有银行}{}{前端开发}{移动端开发}{
	\begin{tightitemize}
		\item 项目描述: 某大型国有银行同业金融合作平台。
		\item 语言: JavaScript、Java
		\item 技术栈:  Angular.js、Ionic、Cordova
		\item 主要贡献: 1. 负责开发网站前端开发。 2. 开发基于混合应用框架Ionic的手机APP。
	\end{tightitemize}}

\section{技能}

\subsection{语言}
\cvcomputer{精通}{JavaScript、Java}
           {熟练}{Python、Ruby}

\subsection{框架}           
\cvcomputer{精通}{Backbone、Spring、Django、Angular 2、Ionic}           
           {熟练}{Angular、React}
		
\section{社区}
\cvitemwithcomment{个人博客}{ \href{https://www.phodal.com/}{https://www.phodal.com}}{日均PV:500}
\cvitemwithcomment{GitHub}{ \href{https://github.com/phodal}{https://github.com/phodal}}{全球Top 100}
\cvitemwithcomment{InfoQ}{ http://www.infoq.com/cn/author/黄峰达 }{社区编辑}
\cvitemwithcomment{CSDN}{ \href{http://blog.csdn.net/phodal}{http://blog.csdn.net/phodal}}{前端博客专家,头条主编}

\section{开源项目}           
\cventry{2015-今}{跨平台Web开发学习应用—Growth(可从App Store或Google Play、应用宝等Android应用商店下载)}{}{}{}{日活 300}

\section{开源电子书}           

\cventry{2014}{《教你设计物联网系统》GitHub \href{https://github.com/phodal/designiot}{https://github.com/phodal/designiot} }{}{}{}{}
\cventry{2015}{《GitHub漫游指南》GitHub \href{https://github.com/phodal/github-roam}{https://github.com/phodal/github-roam} }{}{}{}{}
\cventry{2015-2016}{《全栈增长工程师指南》GitHub \href{https://github.com/phodal/growth-ebook}{https://github.com/phodal/growth-ebook} }{}{}{}{}
\cventry{2016}{《全栈增长工程师实战》GitHub \href{https://github.com/phodal/growth-in-action}{https://github.com/phodal/growth-in-action} }{}{}{}{}

\section{其他} 
\cventry{电子工业出版社}{《自己动手设计物联网系统》}{作者}{}{}{}
\cventry{PACKT 出版社}{《Learning Internet of Things》、《Smart IoT Projects》}{英文版技术审阅}{}{}{}
\cventry{机械工业出版社}{《物联网实战指南》}{中文版译者}{}{}{}

\section{{教育背景}}
\cventry{2010.9-2014.6}{{本科}}{电子信息工程}{西安文理学院}{}{}

\renewcommand{\listitemsymbol}{-}             % 改变列表符号

% 来自BibTeX文件但不使用multibib包的出版物
%\renewcommand*{\bibliographyitemlabel}{\@biblabel{\arabic{enumiv}}}% BibTeX的数字标签
\nocite{*}
\bibliographystyle{plain}
\bibliography{publications}                    % 'publications' 是BibTeX文件的文件名

% 来自BibTeX文件并使用multibib包的出版物
%\section{出版物}
%\nocitebook{book1,book2}
%\bibliographystylebook{plain}
%\bibliographybook{publications}               % 'publications' 是BibTeX文件的文件名
%\nocitemisc{misc1,misc2,misc3}
%\bibliographystylemisc{plain}
%\bibliographymisc{publications}               % 'publications' 是BibTeX文件的文件名

\clearpage\end{CJK}
\end{document}


%% 文件结尾 `template-zh.tex'.
