%% start of file `template-zh.tex'.
%% Copyright 2006-2013 Xavier Danaux (xdanaux@gmail.com).
%
% This work may be distributed and/or modified under the
% conditions of the LaTeX Project Public License version 1.3c,
% available at http://www.latex-project.org/lppl/.


\documentclass[11pt,a4paper,sans]{moderncv}  
\usepackage[utf8]{inputenc}                  
\usepackage{CJKutf8}
\usepackage{info}
% moderncv 主题
                     

\setlength{\hintscolumnwidth}{3cm}           % 如果你希望改变日期栏的宽度
 

% 个人信息
\name{Huang}{Fengda}

%\title{简历题目 (可选项)}                     % 可选项、如不需要可删除本行
\address{ShaanXi Provice, Xi'an City,Yanta qu,Ganjiazhai 31 haolou 3 danyuan}{710065}            % 可选项、如不需要可删除本行
\phone[mobile]{+18209219631}              % 可选项、如不需要可删除本行
% 可选项、如不需要可删除本行
\email{h@phodal.com}                    % 可选项、如不需要可删除本行
\homepage{www.phodal.com}                  % 可选项、如不需要可删除本行
\extrainfo{Power by \LaTeX  }                 % 可选项、如不需要可删除本行
\photo[64pt][0.2pt]{picture}                  % ‘64pt’是图片必须压缩至的高度、‘0.4pt‘是图片边框的宽度 (如不需要可调节至0pt)、’picture‘ 是图片文件的名字;可选项、如不需要可删除本行

\begin{document}
\begin{CJK}{UTF8}{gbsn}                       % 详情参阅CJK文件包
\maketitle

\section{{Education}}
\cventry{2010.09--2014.06}{Bachelor of Electronic and Information Engineering}{}{Xi'an City}{}{Xi'an University}  % arguments 3


\subsection{{Behavior on school}}

\tllabelcventry{2012.8}{2012.12}{2012.12}{Challenge Cup make competition of Xi'an University special-alass award}{}{}{}{
\begin{tightitemize}
	\item name:《Basis on Android and Arduino Home Automation》 
	\item works: Android client app,server client app,website design,
\end{tightitemize}}

\tllabelcventry{2012.10}{2013.6}{2013.6}{Challenge Cup  of ShannXi Province Technology and Science second prize}{}{}{}{
\begin{tightitemize}
	\item name:《Remote Control's Cloud Home Automation》 
	\item works: Android  client app,server client app,write document
\end{tightitemize}}

\tllabelcventry{2013.1}{2013.7}{2013.7}{Freescale Cup of China's west zone third prize}{}{}{}{}

\section{Computer Skills}
\subsection{Dev}
\cvcomputer{Framework}{Django, jQuery,BootStrap,Laravel,Nodejs,Spring}
           {Language}{JavaScript,Java,Python,C,Ruby}

\cvcomputer{Office}{LibreOffice, Microsoft Office}
           {OS}{GNU/Linux(openSUSE, Centos),MacOS}
\cvcomputer{Design}{ Photoshop, GIMP}
           {Editor}{Emacs, VIM, Sublime Text,Eclipse,Intellij IDEA}

\section{Experience}
\subsection{Web Design}
\tllabelcventry{2012.6}{2012.10}{2012.10}{Fujian Rongyu Company}{Website Upgrade}{}{}{
\begin{tightitemize}
        \item Website Front and Ops
        \item Website: {\httplink{www.fjrongyu.com/}}
\end{tightitemize}}
\tllabelcventry{2013.7}{2013.9}{2013.7}{Shaanxi sheng Fujian Shanghui}{Website Migrate}{}{}{
\begin{tightitemize}
        \item Website Re-Design
        \item website: {\httplink{www.sxsfjsh.com/}}
\end{tightitemize}}
\section{{Experience}}
\tllabelcventry{2012}{2013.4}{}{{Android Client of Remote Control's Home Automation}}{Main Programmer}{}{}{
\begin{tightitemize}
        \item Keywords: RESTful,Java
        \item communication Android with MCU
        \item Homepage: {\httplink{code.google.com/p/home-anywhere/}}
\end{tightitemize}}
\tllabelcventry{2012}{0}{}{{Remote Control's Cloud Home Automation}}{Manager}{}{}{
\begin{tightitemize}
        \item Keywords:HTTP,RESTful,Ajax
		\item Language:Bash,Javascript,Python
        \item Use Ajax come true real time data
        \item Homepage: {\httplink{b.phodal.com}}
\end{tightitemize}}

\subsection{Others}
\tlcventry{2010.12}{0}{Others}{}{}{}{
\begin{tightitemize}
 \item use jQuery and Timerliner build resume {\httplink{about.phodal.com}}
 \item a CMS by Laravel {\httplink{www.xianuniversity.com}}
 \item use Python,Django,Bootstrap build person homepage -- {\httplink{www.phodal.com}}
 \item use Freescale K60 do a smart car
 \item use LFS compile person GNU/Linux
 \item compile GCC for ARM on Android G1-- {\httplink{blog.csdn.net/phodal}}
 \item use OpenWRT run RESTful of Python
\end{tightitemize}}

\section{Info}
\cvitem{Interests}{\small Compute,Internet,Website Design,Embedded OS,Framework Design,DIY,Design,Open Source}
\cvitem{Tech blog}{\small 
 {\httplink{http://blog.csdn.net/phodal}}}
\renewcommand{\listitemsymbol}{-}             % 改变列表符号
\clearpage\end{CJK}
\end{document}


%% 文件结尾 `template-zh.tex'.
